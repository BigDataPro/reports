\documentclass{article}\usepackage{graphicx, color}
%% maxwidth is the original width if it is less than linewidth
%% otherwise use linewidth (to make sure the graphics do not exceed the margin)
\makeatletter
\def\maxwidth{ %
  \ifdim\Gin@nat@width>\linewidth
    \linewidth
  \else
    \Gin@nat@width
  \fi
}
\makeatother

\definecolor{fgcolor}{rgb}{0.2, 0.2, 0.2}
\newcommand{\hlnumber}[1]{\textcolor[rgb]{0,0,0}{#1}}%
\newcommand{\hlfunctioncall}[1]{\textcolor[rgb]{0.501960784313725,0,0.329411764705882}{\textbf{#1}}}%
\newcommand{\hlstring}[1]{\textcolor[rgb]{0.6,0.6,1}{#1}}%
\newcommand{\hlkeyword}[1]{\textcolor[rgb]{0,0,0}{\textbf{#1}}}%
\newcommand{\hlargument}[1]{\textcolor[rgb]{0.690196078431373,0.250980392156863,0.0196078431372549}{#1}}%
\newcommand{\hlcomment}[1]{\textcolor[rgb]{0.180392156862745,0.6,0.341176470588235}{#1}}%
\newcommand{\hlroxygencomment}[1]{\textcolor[rgb]{0.43921568627451,0.47843137254902,0.701960784313725}{#1}}%
\newcommand{\hlformalargs}[1]{\textcolor[rgb]{0.690196078431373,0.250980392156863,0.0196078431372549}{#1}}%
\newcommand{\hleqformalargs}[1]{\textcolor[rgb]{0.690196078431373,0.250980392156863,0.0196078431372549}{#1}}%
\newcommand{\hlassignement}[1]{\textcolor[rgb]{0,0,0}{\textbf{#1}}}%
\newcommand{\hlpackage}[1]{\textcolor[rgb]{0.588235294117647,0.709803921568627,0.145098039215686}{#1}}%
\newcommand{\hlslot}[1]{\textit{#1}}%
\newcommand{\hlsymbol}[1]{\textcolor[rgb]{0,0,0}{#1}}%
\newcommand{\hlprompt}[1]{\textcolor[rgb]{0.2,0.2,0.2}{#1}}%

\usepackage{framed}
\makeatletter
\newenvironment{kframe}{%
 \def\at@end@of@kframe{}%
 \ifinner\ifhmode%
  \def\at@end@of@kframe{\end{minipage}}%
  \begin{minipage}{\columnwidth}%
 \fi\fi%
 \def\FrameCommand##1{\hskip\@totalleftmargin \hskip-\fboxsep
 \colorbox{shadecolor}{##1}\hskip-\fboxsep
     % There is no \\@totalrightmargin, so:
     \hskip-\linewidth \hskip-\@totalleftmargin \hskip\columnwidth}%
 \MakeFramed {\advance\hsize-\width
   \@totalleftmargin\z@ \linewidth\hsize
   \@setminipage}}%
 {\par\unskip\endMakeFramed%
 \at@end@of@kframe}
\makeatother

\definecolor{shadecolor}{rgb}{.97, .97, .97}
\definecolor{messagecolor}{rgb}{0, 0, 0}
\definecolor{warningcolor}{rgb}{1, 0, 1}
\definecolor{errorcolor}{rgb}{1, 0, 0}
\newenvironment{knitrout}{}{} % an empty environment to be redefined in TeX

\usepackage{alltt}
\usepackage[sc]{mathpazo}
\usepackage{geometry}
\geometry{verbose,tmargin=1cm,bmargin=2.5cm,lmargin=2.5cm,rmargin=2.5cm}
\setcounter{secnumdepth}{2}
\setcounter{tocdepth}{2}
\usepackage{url}
\usepackage{hyperref}
\usepackage{times}
\usepackage{textcomp}
\IfFileExists{upquote.sty}{\usepackage{upquote}}{}

\begin{document}

\title{Reports Directory and Work Flow}
\author{}
\date{}
\maketitle

\vspace{-1cm}
The project template includes these main directories and scripts:

\begin{enumerate}
  \item \textbf{ARTICLES} - A directory to store related articles:
  \begin{enumerate}
    \item \textbf{website.txt} a txt file to store websites related to the project
    \item \textbf{notes\_xxx.xlsx/notes\_xxx.csv} a speadsheet to store notes on the articles; the \textbf{QQ} function is useful to import PDF text; the \href{http://trinker.github.io/reports/cite.html}{cite family} of functions are useful for formatting quoted text from the notes\_xxx document (where xxx is the report name)
  \end{enumerate}
  \item \textbf{OUTLINE} - A directory containing the following .tex files:
  \begin{enumerate}
    \item \textbf{outline.tex} an outline skeletal .tex file to outline the structure of  
    \item \textbf{preamble2.tex} a generic preamble .tex file used by outline.tex  
  \end{enumerate}  
  \item \textbf{PRESENTATION} - A directory containing one or more of the following: .Rnw, .pptx and/or .Rmd files:
  \begin{enumerate}
    \item \textbf{xxx.Rmd} a skeletal .Rmd file to create .md files using the \textbf{knit HTML} button.  After running this using the \textbf{html5} function will generate an html5 set of slides.
    \item \textbf{xxx.Rnw} a skeletal .Rnw file to create a set of \href{http://www.math.umbc.edu/~rouben/beamer/}{beamer} slides.  
    \item \textbf{xxx.pptx}  skeletal .pptx slides.      
     \item \textbf{figure} a directory to store figures for easy sourcing
  \end{enumerate} 
  \item \textbf{REPORT} - A directory containing the following files:
  \begin{enumerate}
    \item \textbf{xxx.bib} an optional .bib file supplied to the \textbf{bib.loc} argument of the \textbf{new\_report} function
    \item \textbf{preamble.tex} a generic apa6 preamble .tex file used by xxx.tex (generated for tex templates only) 
    \item \textbf{xxx.tex}, \textbf{xxx.Rnw}, \textbf{xxx.Rmd} or \textbf{xxx.doc} a skeletal file used to generate a formatted report
    \item \textbf{DESCRIPTION} description of the template used
    \item \textbf{figure} an optional directory to store figures for easy sourcing
  \end{enumerate} 
  \item \textbf{.Rprofile} - An .Rprofile that automatically launches commands, loads libraries and sets options on start up.
  \item \textbf{extra\_functions.R} - An R script to put functions related to the report.  By default this is sourced on start up in RStudio.  
  \item \textbf{NOTES.txt} - a .txt file to record notes related to the report
  \item \textbf{REPORT\_WORKFLOW\_GUIDE.Rnw} - The workflow guide
  \item \textbf{xxx.Rproj} - A project file used by \href{http://www.rstudio.com/}{RStudio}; clicking this will open the project in RStudio. 
  \item \textbf{TO\_DO.txt} - A text file documenting project tasks
\end{enumerate}
\vspace{1.5 mm}
{\scriptsize \textbf{\emph{*Note: additional files/directores may be created, varying by template. See \texttildelow report\_name/REPORT/DESCRIPTION for more on individual templates.}}}
\vspace{3 mm}
\hrule
\vspace{3 mm}
\noindent \textbf{\underline{Useful Websites:}} \vspace{1.5 mm} \\ 
\textbf{reports Package} - \href{https://github.com/trinker/reports}{https://github.com/trinker/reports}\\ 
\textbf{RStudio} - \href{http://www.rstudio.com/}{http://www.rstudio.com/}\\
\textbf{knitr} - \href{http://yihui.name/knitr/}{http://yihui.name/knitr/}\\
\textbf{Beamer} - \url{http://www.math.umbc.edu/~rouben/beamer/}\\
\textbf{Beamer Themes} - \url{http://deic.uab.es/~iblanes/beamer_gallery/}\\
\textbf{slidify} - \url{http://ramnathv.github.io/slidify/}\\    
\textbf{dzslides} - \url{http://paulrouget.com/dzslides/}\\
\textbf{reveal.js} - \url{http://lab.hakim.se/reveal-js/#/}\\
\end{document}









